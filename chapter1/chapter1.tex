%%%%%%%%%%%%%%%%%%%%
% Introduction chapter

Text

\section{Historical overview of meiotic recombination}
%Brief historical review of meiotic recombination, pre Human Genome Project.

\section{Current status of recombination}
% Current status of recombination field in humans
\begin{titemize}
    \item Current ``gold standard'' maps (Hapmap2 LD map, deCODE pedigree map).
    \item Sexual dimorphism in recombination.
    \item Determinants of recombination placement (PRDM9, GC, chromosome position).
    \item Maternal age effect.
    \item Recombination in other organisms, including dogs.
\end{titemize}
\section{Crossover interference}
% Current status of crossover interference field.
\begin{titemize}
    \item Two-pathway model.
    \item Cover what's known in humans, dogs, other organisms.
\end{titemize}
\section{Hypothesis and goals}
% Hypothesis / goal of this thesis.
\begin{titemize}
    \item To investigate specific properties of recombination and how it differs between sexes, individuals, populations, and species.
        %\item To examine how the absence of PRDM9 has affected the recombination landscape in dogs.
    \item To examine how crossover interference varies between individuals and across species.
    \item To further investigate possibilities to model gene conversion in the human genome.
\end{titemize}
\section{Description of approach}
% Description of approach.
\begin{titemize}
    \item Pedigree analysis vs LD vs experimental methods to detect recombination.
    \item Models of crossover interference.
    \item Gene conversion in admixed populations HMM \& statistical methods for recombination modeling.
\end{titemize}


