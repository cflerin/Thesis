%%%%%%%%%%%%%%%%%%%%
% crossover interference manuscript chapter

% \DoubleSpacing

\begin{abstract}

Recombination plays a fundamental role in meiosis, ensuring the proper segregation of
chromosomes and contributing to genetic diversity by generating novel combinations of
alleles. Here, we use data derived from direct-to-consumer genetic testing to investigate
patterns of recombination in over 4,200 families. Our analysis reveals a number of sex
differences in the distribution of recombination. We find the fraction of male events occurring
within hotspots to be 4.6% higher than for females. We confirm that the recombination rate
increases with maternal age, while hotspot usage decreases, with no such effects observed in
males. Finally, we show that the placement of female recombination events appears to
become increasingly deregulated with maternal age, with an increasing fraction of events
observed within closer proximity to each other than would be expected under simple models
of crossover interference.

\end{abstract}


\section{Introduction}

Recombination is a fundamental meiotic process that is
required to ensure the proper segregation of chromosomes.
In mammals and other eukaryotes, at least one crossover is
normally required to ensure proper disjunction, and failures in
recombination can result in deleterious outcomes such as
aneuploidy. As such, the recombination process is highly
regulated to ensure that sufficient numbers of crossovers occur.
The placement of crossover events along a chromosome is also
tightly regulated. At the fine scale, the majority of crossovers tend
to occur within localized regions of $\sim$2 kb in width known as
recombination hotspots. At broader scales, interference between
crossovers appears to increase spacing between events occurring
on the same chromosome during meiosis.

As relatively few crossover events occur within a single meiosis,
quantifying the recombination landscape requires the observation
of large numbers of meioses. In this study, we adopt a pedigree
approach to study the properties of recombination in over 18,000
meioses using data derived from families genotyped via 
direct-to-consumer genetic testing. Our approach enables hundreds of
thousands of recombination events to be localized, and allows us
to investigate how the frequency and placement of recombination
changes as a function of sex and parental age.

\section{Methods and Materials}

\section{Results}

To investigate properties of crossover placement in humans, we
collected data from pedigree families contained within the
database of 23andMe Inc. (Mountain View, CA). Our data set
consists of 4,209 families contributing a total of 18,302
informative meioses genotyped at over 515,972 sites. To preserve
the privacy of the participants, families were removed if the age of
the mother was greater than 40 years at the time of childbirth,
the age of the father was greater than 45 years or the difference
between the parental ages was greater than 15 years
(Supplementary Fig. 1). The majority of the data is derived from
family quartets (Supplementary Table 1), accounting for 78.6\% of
the families, and is also predominately composed of individuals of
European ancestry (Supplementary Table 2). Ancestral populations
are assigned to each individual by comparison with a set of
reference populations (see Supplementary Methods).

To infer recombination events in nuclear families, we applied
the Lander–Green algorithm as implemented within \citet{Abecasis2002}.
To guard against genotyping error, we curated the data to remove
nearby recombination events that could be indicative of
genotyping error (see Supplementary Methods; Supplementary
Fig. 2). This approach allowed us to identify over 645,000 
well-supported crossover events, with the median event being localized
to 28.2 kb (Supplementary Fig. 3).

We inferred a mean of 41.6 autosomal recombination events
per gamete in females (95\% confidence interval (CI): 41.4–41.9)
and 26.6 in males (95\% CI: 26.5–26.7, Fig. 1a). The genetic map
constructed from our data agrees well with those generated by
previous studies (Fig. 1b; Supplementary Fig. 4; Supplementary
Table 3). At the 5-Mb scale, the Pearson correlation between our
map and that of deCODE\cite{Kong2010} is $r^2$ = 0.975 and 0.983 for females and
males, respectively. Likewise, our sex-averaged map has a
correlation of $r^2$ = 0.955 with the HapMap map inferred from
patterns of linkage disequilibrium (LD)\cite{hapmap2007}. At the chromosome
scale, the map length is well predicated by the physical
chromosome length ($r^2$ = 0.991 in females and 0.945 in males;
Supplementary Fig. 5).

Treating the overall recombination rate as a phenotype, we
replicate genetic associations at genome-wide significance for
RNF212, which is known to be essential for crossover-specific
complexes\cite{Reynolds2013}, and within the vicinity of TTC5, which appears
to replicate an association with CCNB1IP1 (\citet{Kong2014}). Another

association near SMEK1 also replicates discoveries elsewhere\cite{Kong2014},
but not at genome-wide significance (Supplementary Table 4).

Previous reports have suggested increased recombination rates
in older females\cite{Kong2004,Coop2008}. Using linear regression (Supplementary
Fig. 6), we obtain a similar result with an additional 0.067
events per year being observed in females (P = 0.002, F-test), and
no such effect being observed in males (P = 0.30, F-test). The
female effect appears to be driven by sharp increase in the
number of recombination events for older mothers (Fig. 1c).
Fitting the piecewise-linear model with a single change point
infers a rapid increase in the female recombination rate after 38.8
years, increasing from 0.047 events per year to 2.990 events per
year. On average, mothers of 39 years and over have an additional
2.51 events compared with younger mothers (P = 0.0005,
Mann–Whitney U).

One possible interpretation of the increasing number of
recombination events with maternal age is that mothers with
higher recombination rates can maintain fertility until a later
age\cite{Kong2004} . To investigate this possibility, we focused on 776 mothers
(providing 2,184 meioses) that were part of larger families and
could have recombination events assigned to specific children.
After subtracting off the average age and average number of
recombination events for each mother, the resulting regression
does not find a significant association with age (P = 0.11, F-test),
although we estimate our power to detect an effect size of an
additional 0.067 events per year in this subsample to be no more
than 30%.

Both pedigree and LD studies have suggested that $\sim$60–70\% of
crossover events occur within recombination hotspots\cite{Coop2008,Myers2005}. Our
data confirm this result with 62.7\% of events occurring within
LD-defined hotspots in females, and 67.3\% occurring within
hotspots in males (Fig. 2a; Supplementary Fig. 7A). The 4.6\%
difference between the two sexes is highly significant
($P = 1.1 \times 10^{-69}$, Mann–Whitney U), suggesting differences in
the regulation of crossover placement between the sexes. The
result remains significant after thinning the female data to match
the crossover density of the male data ($P < 2.2 \times 10^{-16}$,
Mann–Whitney U), and does not appear to be driven by
increased male recombination rates near the telomeres (see
Supplementary Methods).

Hotspot localization is believed to be under the control of the
zinc-finger protein PRDM9, which recognizes and binds specific
DNA motifs\cite{Berg2010,Berg2011,Hinch2011,Parvanov2010}. We find single-nucleotide polymorphisms
(SNPs) in the vicinity of PRDM9 to be strongly associated with
the degree of hotspot usage, as has previously been reported\cite{Kong2014,Hinch2011}.
The most strongly associated SNP is rs73742307 achieving a
P value of $7.9 \times 10^{-184}$ (\citet{Reynolds2013}), with no other region achieving a
genome-wide significant association with this phenotype
(Supplementary Table 5).

Variation within the PRDM9 DNA-binding domain can result
in changes to the recognized motif and hence lead to differences
in hotspot localization between individuals. While the major allele
of PRDM9 (allele A) is present at high frequency in most human
populations, a large number of low-frequency alleles have been
observed, particularly within African populations\cite{Berg2011,Parvanov2010}. Consistent
with this, we find hotspot usage to be significantly lower within
individuals of African ancestry (Fig. 2b; Supplementary Table 6),
which reflects the fact that the LD-defined hotspots are expected
to mostly represent the common PRDM9 allele. Notably, while
over 75\% of our data are derived from individuals of European
ancestry, hotspot usage is higher for males than females across all
ancestries.

We find a weak association between hotspot usage and
maternal age (Supplementary Fig. 7B). Using logistic regression,
we estimate a decrease in hotspot usage corresponding to
$\sim$1\% over a 10-year period ($\beta_1 = -0.0042$, s.e. = $9.6 \times 10^{-4}$, 




\section{Discussion}


\bibliographystyle{ccampbell_thesis} %unsrtnat} %abbrvnat_noURL} %abbrvUnsrt_last-first} %plain,unsrt,alpha,abbrv,acm,apalike,unsrt
\bibliography{/home/ccampbell/Dropbox/papers/thesis-cointEscape}

\section{Appendix}
