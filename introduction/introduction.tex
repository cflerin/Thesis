%%%%%%%%%%%%%%%%%%%%
% Introduction chapter

%%%%%%%%%%%%%%%%%%%%%%%%%%%%%%%%%%%%%%%%
%%%%%%%%%%%%%%%%%%%%%%%%%%%%%%%%%%%%%%%%
\section{Meiotic recombination overview}
%%%%%%%%%%%%%%%%%%%%%%%%%%%%%%%%%%%%%%%%
%%%%%%%%%%%%%%%%%%%%%%%%%%%%%%%%%%%%%%%%

%%%%%%%%%%%%%%%%%%%%%%%%%%%%%%%%%%%%%%%%
%%%%%%%%%%%%%%%%%%%%%%%%%%%%%%%%%%%%%%%%
\section{Historical overview of meiotic recombination}
%%%%%%%%%%%%%%%%%%%%%%%%%%%%%%%%%%%%%%%%
%%%%%%%%%%%%%%%%%%%%%%%%%%%%%%%%%%%%%%%%

Brief historical review of meiotic recombination, pre Human Genome Project.

%%%%%%%%%%%%%%%%%%%%%%%%%%%%%%%%%%%%%%%%
%%%%%%%%%%%%%%%%%%%%%%%%%%%%%%%%%%%%%%%%
\section{Current status of recombination}
%%%%%%%%%%%%%%%%%%%%%%%%%%%%%%%%%%%%%%%%
%%%%%%%%%%%%%%%%%%%%%%%%%%%%%%%%%%%%%%%%

% Current status of recombination field in humans
\subsection{Methods for studying recombination}
% brief introduction to these before describing in detail in "description of approach"
\subsubsection{Molecular assays}
\paragraph{Sperm cell assays.}
\subsubsection{Linkage disequilibrium}
\subsubsection{Pedigree analysis}

\subsection{Current ``gold standard'' maps (Hapmap2 LD map, deCODE pedigree map).}

\subsection{Sexual dimorphism in recombination.}
\paragraph{Heterochiasmy.}


\subsection{Genes involved in recombination (RFN212, etc)}
Reynolds2013

\subsection{Determinants of recombination placement (PRDM9, GC, chromosome position).}

Differences in populations (Berg2010,Berg2011,Hinch2011)

\subsection{Maternal age effect.}

Kong2004, Coop2008

Hussin2011, Martin2015

Polani1991, Rowsey2014: Production line hyp.

\subsection{Recombination in other organisms, including dogs.}
\subsection{Heritance of recombination modifiers (rate, etc. Kong)}
\subsection{Sexual dimorphism in recombination rate}

Haldane-Huxley rule: when recombination is absent in one sex, it is the heterogametic sex.

Trivers hypothesis: recombination is lower in the sex that undergoes stronger selection (recombination disrupts favorable haplotypes in the most fit individuals, therefore is selected against).


\subsection{Hotspots}

\subsubsection{Conservation between species}
Humans and chimpanzees have a complete absence of hotspot sharing, despite a high degree of overall DNA sequence identity\cite{Ptak2005,Winckler2005,Auton2012a}.

Discuss hotspot paradox vs stable hotspots theory (latter saying hotspots are confined to specific chromosome features (promoters/GC content), as in yeast and dogs)
\subsubsection{Species lacking PRDM9}
Dogs, birds (Singhal2015, biorxiv) 


%%%%%%%%%%%%%%%%%%%%%%%%%%%%%%%%%%%%%%%%
%%%%%%%%%%%%%%%%%%%%%%%%%%%%%%%%%%%%%%%%
\section{Crossover interference}
%%%%%%%%%%%%%%%%%%%%%%%%%%%%%%%%%%%%%%%%
%%%%%%%%%%%%%%%%%%%%%%%%%%%%%%%%%%%%%%%%

% Current status of crossover interference field.
\begin{titemize}
    \item gamma and two-pathway model, math review.
    \item other methods of measuring coint
    \item genetic map functions taking into account coint Speed,Zhao
    \item potential method of coint action.
    \item Cover what's known in humans, dogs, other organisms.
\end{titemize}

%%%%%%%%%%%%%%%%%%%%%%%%%%%%%%%%%%%%%%%%
%%%%%%%%%%%%%%%%%%%%%%%%%%%%%%%%%%%%%%%%
\section{Gene conversion}
%%%%%%%%%%%%%%%%%%%%%%%%%%%%%%%%%%%%%%%%
%%%%%%%%%%%%%%%%%%%%%%%%%%%%%%%%%%%%%%%%

\begin{titemize}
    \item Review of Li \& Stephens and other models
    \item HMM
\end{titemize}

%%%%%%%%%%%%%%%%%%%%%%%%%%%%%%%%%%%%%%%%
%%%%%%%%%%%%%%%%%%%%%%%%%%%%%%%%%%%%%%%%
\section{Hypothesis and goals}
%%%%%%%%%%%%%%%%%%%%%%%%%%%%%%%%%%%%%%%%
%%%%%%%%%%%%%%%%%%%%%%%%%%%%%%%%%%%%%%%%

% Hypothesis / goal of this thesis.
\begin{titemize}
    \item To investigate specific properties of recombination and how it differs between sexes, individuals, populations, and species.
        %\item To examine how the absence of PRDM9 has affected the recombination landscape in dogs.
    \item To examine how crossover interference varies between individuals and across species.
    \item To further investigate possibilities to model gene conversion in the human genome.
\end{titemize}

%%%%%%%%%%%%%%%%%%%%%%%%%%%%%%%%%%%%%%%%
%%%%%%%%%%%%%%%%%%%%%%%%%%%%%%%%%%%%%%%%
\section{Description of approach}
%%%%%%%%%%%%%%%%%%%%%%%%%%%%%%%%%%%%%%%%
%%%%%%%%%%%%%%%%%%%%%%%%%%%%%%%%%%%%%%%%

% Description of approach.
\subsection{Methods used here to study recombination}
% brief introduction to these before describing in detail in "description of approach"
\subsubsection{Linkage disequilibrium}
\subsubsection{Pedigree analysis}
\paragraph{Lander-Green algorithm.}
\paragraph{duoHMM.}

\begin{titemize}
    \item Models of crossover interference (described in full in nat comm paper).
    \item Gene conversion in admixed populations HMM \& statistical methods for recombination modeling.
\end{titemize}



%%%%%%%%%%%%%%%%%%%%%%%%%%%%%%%%%%%%%%%%
%%%%%%%%%%%%%%%%%%%%%%%%%%%%%%%%%%%%%%%%
\bibliographystyle{ccampbell_thesis} %unsrtnat} %abbrvnat_noURL} %abbrvUnsrt_last-first} %plain,unsrt,alpha,abbrv,acm,apalike,unsrt
\bibliography{/home/ccampbell/Dropbox/papers/recombination}
%%%%%%%%%%%%%%%%%%%%%%%%%%%%%%%%%%%%%%%%
%%%%%%%%%%%%%%%%%%%%%%%%%%%%%%%%%%%%%%%%

